\chapter{Literature Survey}

\begin{Large}
\textbf{Review of Literature}\\
\end{Large}
\vspace*{0.5cm}\\
\begin{enumerate}
	\item	
		\textbf{Title}
			\paragraph{}Machine Learning\\
		
		\textbf{Published by}
			\paragraph{}Panos Luridas, Christof Ebert, 2016
			\paragraph{}Machine learning is the major success factor in the ongoing digital transformation across industries. The general idea behind most machine learning is that a computer learns to perform a task by studying a training set of examples. The computer (or system of distributed or embedded computers and controllers) then performs the same task with data it hasn’t encountered before.\\

	\item
		\textbf{Title}
			\paragraph{}Activity Discovery and Detection of Behavioral Deviations\pagebreak

		\textbf{Published by}
			\paragraph{}Jeremie Saives, Clement Pianon, and Gregory Faraut, 2016
			\paragraph{}The aim of this system is to improve the autonomy of medically monitored patients in a smart home instrumented only with binary sensors over watching the disease evolution, that can be characterized by behavior changes, is helped by detecting the activities the inhabitant performs. Two contributions are pre-sented. On one hand, using sequence mining methods in the flow of sensor events, the most frequent patterns mirroring activities of the inhabitant are discovered; these activities are then modeled by an extended finite automaton, which can then be used for activity recognition and generate activity events. On the other hand, given the set of activities that can be recognized, another automaton is built to model requirements from the medical staff supervising the inhabitant; it accepts activity events, and residuals are defined to detect any behavior deviation. The whole method is applied to the dataset of Domus, an instrumented smart home.\\	
					
	\item
		\textbf{Title}
		\paragraph{}Anomalies detection in smart-home activities\\
	
		\textbf{Published by}
			\paragraph{}Labiba Gillani Fahad \& Muttukrishnan Rajarajan, 2015
			\paragraph{}Anomalies are the instances of an activity class that deviate from the normal or expected sequence of events in their performance. In order to detect anomalies, we exploit the information of number of events and the time duration involved in performing an activity instance. The information is obtained through a network of wireless sensor deployed at multiple objects and locations within a smart home. They apply a density based clustering algorithm on the recognized activity instances to separate the normal from the anomalous. Evaluation of the proposed approach on two publicly available smart home datasets demonstrates its effectiveness in identifying the anomalous activity instances.\\

	\item
		\textbf{Title}
			\paragraph{}Composite activity recognition in smart homes using Markov Logic Network\\
		
		\textbf{Published by}
			\paragraph{}K.S.Gayathri, Susan Elias, S.Shivashankar, 2015
			\paragraph{}Smart environments have progressed and evolved into a significant research area with development of sensor technology, wireless communication and machine learning strategies. Ambient Intelligence incorporated into smart environment assists in resolving many social related applications to facilitate. The proposed system performs activity modeling via Markov Logic Network, a machine learning strategy that combines probabilistic reasoning and logical reasoning with a single framework. Activities in a smart home are categorized as simple and composite activities, wherein composite activities are defined as related simple activities within a given time interval. The proposed system models both simple and composite activity using soft and hard rules of MLN. Experiments carried over the proposed system shows the effectiveness of the proposed work for recognizing simple and composite activity.\\
			
	\item
		\textbf{Title}
			\paragraph{}Activity Recognition Based on Streaming Sensor Data for Assisted Living in Smart Homes\\
		
		\textbf{Published by}
			\paragraph{}Vahid Ghasemi, Ali Akbar Pouyan, 2015
			\paragraph{}Human activity recognition (HAR) is a fundamental task in smart homes. In these environments residents’ data are collected via unobtrusive sensors, and human activities are inferred using machine learning mechanisms out of sensors’ data. Dynamic graphical models (DGMs) have been a widely used family of machine learning mechanisms for HAR. In DGM-based HAR methods relative temporal information and duration of activities are intrinsically hired for modelling activities, while neglecting absolute temporal information in their primitive forms. Some human activities in the home almost have certain temporal patterns i.e. they take place at certain points of time throughout a periodic time span. Such temporal information can improve the recognition efficiency.\\

	\item
		\textbf{Title}
			\paragraph{}User Activity Recognition for Energy Saving in Smart Home Environment\\
		
		\textbf{Published by}
			\paragraph{}Wesllen S. Lima, Eduardo Souto, Richard W. Pazzi, Ferry Pramudianto, 2015
			\paragraph{}In recent years, the consumption of electricity has increased considerably in the industrial, commercial and residential sectors. This has prompted a branch of research which attempts to overcome this problem by applying different information and communication technologies, turning houses and buildings into smart environments. In this system, they propose and design an energy saving technique based on the relationship between the user's activities and electrical appliances in smart home environments. The proposed method utilizes machine learning techniques to automatically recognize the user's activities, and then a ranking algorithm is applied to relate activities and existing home appliances.\\

	\item
		\textbf{Title}
			\paragraph{}Design and Implementation of an Autonomous Wireless Sensor-based Smart Home\\
		
		\textbf{Published by}
			\paragraph{}Christopher Osiegbu, Seifemichael B. Amsalu, Fatemeh Afghah, Daniel Limbrick and Abdollah Homaifar, 2015
			\paragraph{}The Smart home has gained widespread attention due to its flexible integration into everyday life. This next generation green home system, transparently unifies various home appliances, smart sensors and wireless communication technologies. It can integrate diversified physical sensed information and control various consumer home devices, with the support of active sensor networks having both sensor and actuator components. Although smart homes are gaining popularity due to their energy saving and better living benefits, there is no standardized design for smart homes. In this system, they put forward a concept by designing and implementing a smart home system which can classify and predict the state of the home based on historical data. They set up a wireless sensor network and collected months of data.\\

	\item
		\textbf{Title}
			\paragraph{}A Comparison of Classifiers for Intelligent Machine Usage Prediction\\
		
		\textbf{Published by}
			\paragraph{}Chiming Chang, Paul-Armand Verhaegen, Joost R. Duflou, 2014
			\paragraph{}Probability estimation of machine usages is an essential task to the development of an intelligent device/environment. In this system, they propose a generic framework to the task using the sliding window technique and incremental feature selection. The methodology is applied to a real-life dataset of office printers and the performances of different standard classifiers in supervised learning are compared. They conclude that Logistic Regression (LR) outperform other classifiers and is appropriate for the proposed framework. The use of Generic Bayesian Network (GBN) classifier is also promising, if combined with feature reduction methods.\\

	\item
		\textbf{Title}
			\paragraph{}A Rule-based Service Customization Strategy Context-aware Automation for Smart Home\\
		
		\textbf{Published by}
			\paragraph{}Z. Meng, and J. Lu, 2014
			\paragraph{}The continuous technical progress of the smartphone built-in modules and embedded sensing techniques has created chances for context-aware automation and decision support in home environments. Studies in this area mainly focus on feasibility demonstrations of the emerging techniques and system architecture design that are applicable to the different use cases. It lacks service customization strategies tailoring the computing service to proactively satisfy users’ expectations. This investigation aims to chart the challenges to take advantage of the dynamic varying context information, and provide solutions to customize the computing service to the contextual situations. This work presents a rule-based service customization strategy which employs a semantic distance-based rule matching method for context-aware service decision making and a Rough Set Theory-based rule generation method to supervise the service customization. The simulation study reveals the trend of the algorithms in time complexity with the number of rules and context items.\\

	\item
		\textbf{Title}
			\paragraph{}Keeping the Resident in the Loop: Adapting the Smart Home to the User\\

		\textbf{Published by}
			\paragraph{}Parisa Rashidi, Student Member, IEEE, and Diane J. Cook, Fellow, 2009
			\paragraph{}Recent advancements in supporting fields have increased the likelihood that smart-home technologies will become part of our everyday environments. However, many of these technologies are brittle and do not adapt to the user’s explicit or implicit wishes. Here, they introduce CASAS ,an adaptive smart-home system that utilizes machine learning techniques to discover patterns in resident’s daily activities and to generate automation polices that mimic these patterns. Their approach does not make any assumptions about the activity structure or other underlying model parameters but leaves it completely to our algorithms to discover the smart-home resident’s patterns. Another important aspect of CASAS is that it can adapt to changes in the discovered patterns based on the resident implicit and explicit feedback and can automatically update its model to reflect the changes.In this system, they provide a description of the CASAS technologies and the results of experiments performed on both synthetic and real-world data.\\

\end{enumerate}