\chapter{Introduction}

%%%%%%%%%%%%%%%%
\begin{comment}

\section{Background and Recent Research}
\subsection{<any sub section here>}

\subsection{Literature Survey}

\subsubsection{<Sub-subsection title>}
some text\cite{citation-1-name-here}, some more text

\subsubsection{<Sub-subsection title>}
even more text\footnote{<footnote here>}, and even more.

\section{Motivation}

\end{comment}
%%%%%%%%%%%%%%%%

\section{Detailed Problem Definition}
\paragraph{}
Most present-day smart homes use simple reflex agents for automation. Simple reflex agents are non-flexible and can work with limited percepts and hard-coded actuation rules. These rules may not suit all people. This rigidity in usability of present consumer automation systems forms the core of our problem.
\paragraph{}
We intend to develop a software solution to this problem, that is centered around machine learning. This solution will be in the form of a learning agent that learns user habits by observation and anomaly detection.

\section{Brief Description}
\paragraph{}
This project aims to enhance the home automation experience by collecting usage data from the user and applying prediction algorithms on it to predict the next step the user may take. Furthermore, external data will also be collected an correlated with the usage data in order to determine what external conditions may influence the user's behavior. This involves data like weather data and traffic data.

%\section{Problem Definition}
