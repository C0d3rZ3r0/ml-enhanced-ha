\vspace{2in}

\begin{comment}

\newpage
\begin{center}
\begin{Huge}
\textbf{Abstract}
\end{Huge}
\end{center}

\end{comment}

\begin{abstract}

\paragraph{}
Traditional home automation systems are mostly hard-coded or require manual automation plan generation by users. This requires interaction between a control system (an interface) and the user, requiring quite some effort and time to be put in manual planning of the home environment. This project intends to explore applying ML on real-time usage data to generate personalized and time-variant home automation plans. These plans will save the user time and effort, leading to a smoother ML driven home automation experience. We collect streaming usage statistics from smart-home occupants and store it on a centralized server. Simultaneously, we also collect “external” data (which may consist of environmental factors like natural light intensity, wind speed, et cetera) which may influence occupants’ usage behavior. These datasets are combined, with data timestamps as a unique identifying field, into a super-set. It’s then fed into a Machine Learning system to correlate user habits with time of the day and the external factors. The correlation hence established will be updated in real time. This correlation will be in the form of a prediction model that will be used to predict near future values of target devices for the occupant. Hence, by combining real-time usage data from a conventional home automation system and Machine Learning, we will be able to provide smoother and more comfortable environment to the users, as the burden of plan generation will be greatly reduced.

\end{abstract} 
